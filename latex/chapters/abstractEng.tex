{\fontfamily{cmr}\selectfont 

\begin{center}
\centering
\textbf{\Large{Diploma Thesis}}

\vspace{1cm}

\textbf{\Large{Title}}

\vspace{0.5cm}

\textbf{\large{Invariant Features Based on Third Order Correlation - Application on Image Classification}}

\vspace{1cm}

\centering
\textbf{Abstract}
\end{center}

Computer Vision is being used more and more everyday, in our personal life, as well as in a variety of industrial applications. In order to be able to process an image (or a part of it) accurately and fast, one needs a representation of the image, that does not change when the image does. In light of that, plenty image descriptors are proposed and used nowadays. The current diploma thesis attempts to implement an algorithm that produces an image representation, that is invariant to image transformations, such as rotation, scaling and/or translation, as well as noise. This representation plays the role of a local feature, which through the implementation of the Bag of Words model produces a global feature of the image. Taking advantage of third order correlations and their properties, the representation is created and then tested on real images.  A series of experiments are being conducted producing the features of two thousand images, using the proposed representation and the well known and widely used SIFT algorithm. These features are then classified into twenty classes and the results of the classification are analyzed and compared.

\vspace{1.5cm}

\begin{flushright}
Thomai Stathopoulou\\
Information Processing Laboratory\\
Electrical \& Computer Engineering Department\\
Aristotle University of Thessaloniki\\
July, 2014
\end{flushright}
}