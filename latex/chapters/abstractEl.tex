\vspace{0.5cm}
%\centering
\textbf{\Large{Περίληψη}}

\vspace{1cm}

Στη σημερινή εποχή, ο κλάδος της Μηχανικής Όρασης χρησιμοποιείται όλο και περισσότερο σε πολλές εφαρμογές, τόσο σε προσωπικό όσο και βιομηχανικό επίπεδο. Προκειμένου να μπορεί να γίνεται σωστή και γρήγορη επεξεργασία εικόνας (ή μέρους αυτής) έχουν αναπτυχθεί μέχρι σήμερα πολλοί αλγόριθμοι που υπολογίζουν τοπικούς περιγραφείς/αναπαραστάσεις, οι οποίοι να είναι όσο το δυνατόν πιο ανεξάρτητοι από διάφορες μετατροπές που μπορούν να εφαρμοστούν στις εικόνες και να περιέχουν όσο το δυνατόν πιο χρήσιμη πληροφορία. Στα πλαίσια της παρούσας διπλωματικής εργασίας επιχειρείται, με χρήση των τρίτης τάξης συσχετίσεων και των ιδιοτήτων τους, η υλοποίηση αναπαράστασης, ανεξάρτητης της περιστροφής, κλιμάκωσης ή/και μετατόπισης της εικόνας, καθώς και της προσθήκης θορύβου. Η αναπαράσταση αυτή χρησιμοποιείται ως ένας τοπικός περιγραφέας της υφής μιας γειτονιάς της εικόνας και, με χρήση του μοντέλου Bag of Words συντελεί στη δημιουργία ενός ολικού περιγραφέα της εικόνας. Αφού αναλυθούν οι ιδιότητες που χρησιμοποιούνται και ο τρόπος δημιουργίας της αναπαράστασης, εκτελούνται πειράματα σε πραγματικές εικόνες. Στα πλαίσια των πειραμάτων, γίνεται επεξεργασία περίπου δύο χιλιάδων εικόνων, χρησιμοποιώντας την υλοποιημένη προτεινόμενη αναπαράσταση και τον γνωστό αλγόριθμο SIFT και στη συνέχεια εκτελείται ταξινόμηση των εικόνων αυτών σε είκοσι κλάσεις. Τέλος τα αποτελέσματα και των δύο μεθόδων αναλύονται και συγκρίνονται.

