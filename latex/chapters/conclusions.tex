\chapter{Συμπεράσματα}
\paragraph*{}
Συνοψίζοντας, στα πλαίσια της παρούσας εργασίας επιχειρήθηκε η υλοποίηση αναπαράστασης εικόνας, η οποία να παραμένει αναλλοίωτη στις διάφορες μεταβολές που μπορεί να υποστεί η εικόνα, όπως περιστροφή, κλιμάκωση ή μετατόπιση με σκοπό να χρησιμοποιηθεί ως τοπικός περιγραφέας, έτσι ώστε με τη χρήση του μοντέλου Bag of Words, να είναι δυνατή η δημιουργία περιγραφέων ολόκληρης εικόνας. Από δοκιμές που έγιναν σε τεχνητές εικόνες προκύπτει, πως η αναπαράσταση παραμένει σταθερή, εμφανίζοντας μικρές διαφοροποιήσεις, οι οποίες αποδίδονται στην αλλοίωση που υφίσταται η δομή της εικόνας, κατά την εφαρμογή των μεταβολών. Συγκεκριμένα, αν η μεταβολή δεν επιφέρει καμία αλλοίωση στην εικόνα (π.χ. περιστροφή κατά $90\textdegree$), η αναπαράσταση παραμένει ακριβώς ίδια, ενώ στην περίπτωση που η εικόνα αλλοιώνεται, οι διαφοροποιήσεις που προκύπτουν στις αναπαραστάσεις μεταβάλλονται, ανάλογα με το μέγεθος της εικόνας και τις τιμές που έχουν επιλεχθεί για τις παραμέτρους $(\theta_1, \theta_2)$.

\paragraph*{}
Παρόλο που τόσο στα παραδείγματα με τις τεχνητές εικόνες, όσο και στα πειράματα που εκτελέστηκαν, ο προτεινόμενος περιγραφέας σε συνδυασμό με τη χρήση του μοντέλου Bag of Words απέδωσε πολύ καλά, είναι εμφανές από την εκτέλεση και απόδοση των πειραμάτων με χρήση του αλγόριθμου SIFT πως υπάρχουν περιθώρια σημαντικής βελτίωσης. Αυτό αφορά τόσο στην προγραμματιστική υλοποίηση των τοπικών περιγραφέων, όσο και στην απόδοση της ταξινόμησης. Συγκεκριμένα, οι χρόνοι που απαιτούνται για τη δημιουργία των τοπικών περιγραφέων με την προτεινόμενη υλοποίηση είναι τάξεις μεγέθους μεγαλύτεροι από τους χρόνους που απαιτούνται για τη δημιουργία των τοπικών περιγραφέων με τον αλγόριθμο SIFT. Οι χρόνοι αυτοί θα μπορούσαν σίγουρα να μειωθούν αισθητά, αν παραλληλοποιηθεί η διαδικασία παραγωγής τοπικών περιγραφέων όλης της εικόνας, αφού σε κάθε κόμβο εκτελούνται ακριβώς τα ίδια βήματα. Επιπλέον μείωση μπορεί να προκύψει και με βελτιστοποίηση των αλγορίθμων υπολογισμού της συσχέτισης και δημιουργίας της αναπαράστασης.

\paragraph*{}
Προκειμένου να βελτιωθεί η απόδοση στις ταξινομήσεις, υπάρχουν πολλές δοκιμές που μπορούν να γίνουν προκειμένου να βρεθεί η βέλτιστη αναπαράσταση. Αυτό έχει να κάνει κατ' αρχάς με την πληροφορία που περιέχεται στην αναπαράσταση. Όπως αναφέρθηκε στο Κεφ. \ref{chap:theor} στα πλαίσια της παρούσας εργασίας χρησιμοποιείται για την αναπαράσταση μόνο η DC συνιστώσα του μετασχηματισμού Fourier του $\tilde{x_3}$. Θα μπορούσαν να γίνουν δοκιμές κρατώντας περισσότερη πληροφορία του αρχικού σήματος για την τελική αναπαράσταση, ή ακόμα ρυθμίζοντας τα επίπεδα κβάντισης των γωνιών $\theta_1, \theta_2$. Επιπλέον χρειάζεται να εξακριβωθούν οι συνθήκες στις οποίες λειτουργεί βέλτιστα η αναπαράσταση αναφορικά με το μοντέλο Bag of Words. Αυτό σημαίνει πως μπορούν να γίνουν δοκιμές με το μέγεθος της γειτονιάς για την οποία παράγεται ο τοπικός περιγραφέας, ή ακόμα και το βήμα του πλέγματος που εφαρμόζεται στην εικόνα (π.χ. ο αλγόριθμος ColorDescriptor χρησιμοποιεί πλέγμα, στο οποίο οι κόμβοι σε κάθε δεύτερη ξεκινούν κατά $step/2$ πιο μέσα, προκειμένου να ελαχιστοποιηθεί η επικάλυψη των γειτονιών που χρησιμοποιούνται για την παραγωγή του local feature).